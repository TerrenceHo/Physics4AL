%%%%%%%%%%%%%%%%%%%%%%%%%%%%%%%%%%%%%%%%%%%%%%%%%%%%%%%%%%%%%%%%%%%%%
% LaTeX Template: Project Titlepage Modified (v 0.1) by rcx
%
% Original Source: http://www.howtotex.com
% Date: February 2014
% 
% This is a title page template which be used for articles & reports.
% 
% This is the modified version of the original Latex template from
% aforementioned website.
% 
%%%%%%%%%%%%%%%%%%%%%%%%%%%%%%%%%%%%%%%%%%%%%%%%%%%%%%%%%%%%%%%%%%%%%%

\documentclass[11pt]{report}
\usepackage[a4paper]{geometry}
\usepackage[myheadings]{fullpage}
\usepackage{fancyhdr}
\usepackage{lastpage}
\usepackage{graphicx, wrapfig, subcaption, setspace, booktabs}
\usepackage[T1]{fontenc}
\usepackage[font=small, labelfont=bf]{caption}
\usepackage{fourier}
\usepackage[protrusion=true, expansion=true]{microtype}
\usepackage[english]{babel}
\usepackage{sectsty}
\usepackage{url, lipsum}
\usepackage{amsmath}


\newcommand{\HRule}[1]{\rule{\linewidth}{#1}}
\onehalfspacing
\setcounter{tocdepth}{5}
\setcounter{secnumdepth}{5}

%-------------------------------------------------------------------------------
% HEADER & FOOTER
%-------------------------------------------------------------------------------
\pagestyle{fancy}
\fancyhf{}
\setlength\headheight{15pt}
\fancyhead[L]{Terrence Ho}
\fancyhead[R]{Experiment 1}
\fancyfoot[R]{Page \thepage\ of \pageref{LastPage}}
%-------------------------------------------------------------------------------
% TITLE PAGE
%-------------------------------------------------------------------------------

\begin{document}

\title{ \normalsize \textsc{Physics 4AL}
        \\ [2.0cm]
        \HRule{0.5pt} \\
        \LARGE \textbf{\uppercase{Experiment 2: Measurement of g}}
        \HRule{2pt} \\ [0.5cm]
        \vspace*{2\baselineskip}}

\date{}

\author{
        Terrence Ho | ID: 804793446 \\ 
        Date of Lab: April 25th, 2017 \\
        Lab Section: Tuesday, 5 P.M.\\
        T.A.: David Bauer\\
        Lab Partners: Kai Crane}

\maketitle
\tableofcontents
\newpage

%-------------------------------------------------------------------------------
% Section title formatting
\sectionfont{\scshape}
%-------------------------------------------------------------------------------

%-------------------------------------------------------------------------------
% BODY
%-------------------------------------------------------------------------------

\section*{Derivation of Equation 2.1}
\addcontentsline{toc}{section}{Derivation of Equation 2.1}

\indent We first set the velocity \(V_1\) to be the distance \(d\) travelled between 
the first photogate and the second photogate over time \(T_1\).
Similiarly, the velocity \(V_2\) is equal to the distance \(D\) over time 
traveled \(T_2\) between the second photogate and the landing pad.

\[ V_1 = \frac{d}{T_1}   \qquad\text{and}\qquad     V_2 = \frac{D}{T_2} \]

We substitute these velocities into the kintic equation \(V = V_o + g(t) \),
where \(V = V_2\), \(V_o = V_1\), and \(t\) is equal to the average of the two
times, or \(t = \frac{T_1 + T_2}{2} \).

\[ V_2 = V_1 + g\Bigg(\frac{T_1 + T_2}{2}\Bigg) \]

By substituting in the values for \(V_1\) and \(V_2\), we get an equation that
only contains the units that Equation 2.1 contained.

\[ \frac{D}{T_2} = \frac{d}{T_1} + g\Bigg(\frac{T_1 + T_2}{2}\Bigg) \]

By rearranging the equation so that g is isolated, we end up with Equation 2.1.

\[ g = \frac{2}{T_1 + T_2}\Bigg(\frac{D}{T_2} - \frac{d}{T_1}\Bigg) \textrm{ , in
terms of m/s$^2$}\]

\section*{Plots}
\addcontentsline{toc}{section}{Plots/Graphs}

\section*{Data Tables}
\addcontentsline{toc}{section}{Data Tables}

\begin{table}[h]
    \centering
    \begin{tabular}{|c | c | c | c |}
    \hline
    Trial & Photogate Spacing \(d\)(m) & Gap to impact Sensor \(D\)(m) & Measured Acceleration \(g\)(m/s$^2$) \\
    \hline
    1 & 0.080 $\pm$ 0.005 & .455 $\pm$ 0.005 & 10 $\pm$ \\
    \hline
    2 & 0.080 $\pm$ 0.005 & .540 $\pm$ 0.005 & 10 $\pm$ \\
    \hline
    3 & 0.080 $\pm$ 0.005 & .630 $\pm$ 0.005 & 10 $\pm$ \\
    \hline
    4 & 0.080 $\pm$ 0.005 & .270 $\pm$ 0.005 & 10 $\pm$ \\
    \hline
    5 & 0.080 $\pm$ 0.005 & .720 $\pm$ 0.005 & 10 $\pm$ \\
    \hline
    \end{tabular}
    % \caption{Experiment Results and calculated acceleration values.}
\end{table}

\begin{table}[h]
    \centering
    \begin{tabular}{| c | c | c |}
        \hline
        Trial & Gap to impact sensor \(D\)(m) & Uncertainty in measured
        acceleration \(g\)(m/s$^2$) \\
        \hline
    \end{tabular}
    % \caption{Second Table}
\end{table}

\section*{Conclusion}
\addcontentsline{toc}{section}{Conclusion}

\section*{Extra Credit}
\addcontentsline{toc}{section}{Extra Credit}

\section*{Report}
\addcontentsline{toc}{section}{Report}



%-------------------------------------------------------------------------------
% REFERENCES
%-------------------------------------------------------------------------------
\newpage
\section*{References}
\addcontentsline{toc}{section}{References}

Anand, U., 2010. The Elusive Free Radicals, \textit{The Clinical Chemist,} [e-journal] Available at:<\url{http://www.clinchem.org/content/56/10/1649.full.pdf}> [Accessed 2 November 2013]
\newline
\newline

Biology Forums, 2012. \textit{Normal glomerulus. Acute glomerulonephritis.} [online] Available at: <\url{http://biology-forums.com/index.php?action=gallery;sa=view;id=9284}> [Accessed 23 October 2013].
\newline
\newline

Budisavljevic, M., Hodge, L., Barber, K., Fulmer, J., Durazo-Arvizu, R., Self, S., Kuhlmann, M., Raymond, J. and Greene, E., 2003. Oxidative stress in the pathogenesis of experimental mesangial proliferative glomerulonephritis, \textit{American Journal of Physiology - Renal Physiology,} 285(6), pp. 1138-1148.
\newline
\newline

Chien, C., Lee, P., Chen, C., Ma, M., Lai, M. and Hsu, S., 2001. De Novo Demonstration and Co-localization of Free-Radical Production and Apoptosis Formation in Rat Kidney Subjected to Ischemia/Reperfusion, \textit{Journal of the American Society of Nephrology,} 12(5), pp. 973-982.
\newline
\newline

Couser, W., 1993. Pathogenesis of glomerulonephritis, \textit{Kidney International Supplements,} 42, pp. 19-26.
\newline
\newline

De Gasparo, M., 2002. Angiotensin II and nitric oxide interaction, \textit{Heart Failure Reviews,} [e-journal] Available at:<\url{http://www.ncbi.nlm.nih.gov/pubmed/12379820}> [Accessed 26 October 2013]
\newline
\newline

Edinburgh Renal Education Pages, 2012. \textit{Glomerulonephritis} [online] Available at: <\url{http://www.edrep.org/pages/textbook/glomerulonephritis.php}> [Accessed 25 October 2013].
\newline
\newline

Forbes, J., Coughlan, M. and Cooper, M., 2008. Oxidative Stress as a Major Culprit in Kidney Disease in Diabetes, \textit{Diabetes,} 57(6), pp. 1446-1454.
\newline
\newline

Geeky Medics, 2010. \textit{Glomerulonephritis} [online] Available at: <\url{http://geekymedics.com/2010/10/27/glomerulonephritis/}> [Accessed 25 October 2013].
\newline
\newline

Gryglewski, R., Palmer, R., Moncada, S., 1986. Superoxide anion is involved in the break­down of endothelium derived relaxing factor, \textit{Nature,} 320, pp. 454-456.
\newline
\newline

Halliwell, B., 2001. Free Radicals and other reactive species in Disease, \textit{Encyclopedia of Life Sciences,} [e-journal] Available at:<\url{http://web.sls.hw.ac.uk/teaching/level4/bcm1_2/reading/oxidative_stress/files/Oxidative_stress.pdf}> [Accessed 19 October 2013]
\newline
\newline

Huang, H., Patel, P. and Salahudeen, A., 2001. Lazaroid compounds prevent early but not late stages of oxidant-induced cell injury: potential explanation for the lack of efficacy of lazaroids in clinical trials, \textit{Pharmacological Research,} 41(1), pp. 55-61.
\newline
\newline

Klinger, J., Abman, S. and Gladwin, M., 2013. Nitric Oxide Deficiency and Endothelial Dysfunction in Pulmonary Arterial Hypertension, \textit{American Journal of Respiratory and Critical Care Medicine,} 188(6), pp. 639-646.
\newline
\newline

Lindemann, I., Boettcher, J., Oertel, K., Pasternack, R., Heine, A. and Klebe, G. 2012. Inhibitors of Transglutaminase 2: A therapeutic option in celiac disease, \textit{To be Published,} [e-journal + PDB structure] Available at:<\url{http://www.ebi.ac.uk/pdbe-srv/view/entry/3s3s/summary}> [Accessed 24 October 2013]
\newline
\newline

Mayo Clinic, 2011. \textit{Glomerulonephritis} [online] Available at: <\url{http://www.mayoclinic.com/health/glomerulonephritis/DS00503/}> [Accessed 20 October 2013].
\newline
\newline

McCord, J., Roy, R. and Schaffer, S., 1985. Free radicals and myocardial ischemia. The role of xanthine oxidase, \textit{Advances in myocardiology,} [e-journal] Available at:<\url{http://www.ncbi.nlm.nih.gov/pubmed/2982206}> [Accessed 24 October 2013]
\newline
\newline

National Health Service, 2012. \textit{Causes of glomerulonephritis} [online] Available at: <\url{http://www.nhs.uk/Conditions/Glomerulonephritis/Pages/Causes.aspx}> [Accessed 20 October 2013].
\newline
\newline

Niaudet, P., 2013. \textit{Overview of the pathogenesis and causes of glomerulonephritis in children.} [online] Available at: <\url{http://www.uptodate.com/contents/overview-of- \ the-pathogenesis-and-causes-of-glomerulonephritis-in-children}> [Accessed 21 October 2013].
\newline
\newline

Ronco, P., 2013. \textit{Mechanisms of glomerular crescent formation.} [online] Available at: <\url{http://www.uptodate.com/contents/mechanisms-of-glomerular-crescent-formation}> [Accessed 21 October 2013].
\newline
\newline

Rutchik, J., 2013. \textit{Toxic Neuropathy Clinical Presentation.} [online] Available at: <\url{http://emedicine.medscape.com/article/1175276-clinical#a0216}> [Accessed 26 October 2013].
\newline
\newline

R\&D Systems, 2013. \textit{Technical Information. Ischemia/Reperfusion Injury.} [online] Available at: <\url{http://www.rndsystems.com/cb_detail_objectname_SP96_Ischemia.aspx}> [Accessed 28 October 2013].
\newline
\newline

Salahudeen, A., 1999. Free Radicals in Kidney Disease and Transplantation, \textit{Saudi Journal of Kidney Diseases and Transplantation,} 10(2), pp. 137-143.
\newline
\newline

Sarma, A., Mallick, A. and Ghosh, A., 2010. Free Radicals and Their Role in Different Clinical Conditions: An Overview, \textit{International Journal of Pharma Sciences and Research,} 1(3), pp. 182-192.
\newline
\newline

Shah, S., Baliga, R., Rajapurkar, M. and Fonseca, V., 2007. Oxidants in Chronic Kidney Disease, \textit{Journal of the American Society of Nephrology,} 18(1), pp. 16-28.
\newline
\newline

The University of Utah, Unknown. \textit{Glomerulonephritis} [online] Available at: <\url{http://library.med.utah.edu/WebPath/RENAHTML/RENALIDX.html#8}> [Accessed 25 October 2013].
\newline
\newline

Wang, C. and Salahudeen, A., 1994. Cyclosporine nephrotoxicity: attenuation by an antioxidant -inhibitor of lipid peroxidation in-vitro and in-vivo, \textit{Transplantation,} 58, pp. 940-946.
\newline
\newline

Wang, C. and Salahudeen, A., 1995. Lipid peroxidation accompanies cyclosporine nephrotoxicity: effects of vitamin E, \textit{Kidney International,} 47, pp. 927-934.
\newline
\newline

Weiss, S., 1989. Tissue Destruction by Neutrophils, \textit{New England Journal of Medicine,} 320, pp. 365-376.
\newline
\newline


\end{document}

%-------------------------------------------------------------------------------
% SNIPPETS
%-------------------------------------------------------------------------------

%\begin{figure}[!ht]
%    \centering
%    \includegraphics[width=0.8\textwidth]{file_name}
%    \caption{}
%    \centering
%    \label{label:file_name}
%\end{figure}

%\begin{figure}[!ht]
%    \centering
%    \includegraphics[width=0.8\textwidth]{graph}
%    \caption{Blood pressure ranges and associated level of hypertension (American Heart Association, 2013).}
%    \centering
%    \label{label:graph}
%\end{figure}

%\begin{wrapfigure}{r}{0.30\textwidth}
%    \vspace{-40pt}
%    \begin{center}
%        \includegraphics[width=0.29\textwidth]{file_name}
%    \end{center}
%    \vspace{-20pt}
%    \caption{}
%    \label{label:file_name}
%\end{wrapfigure}

%\begin{wrapfigure}{r}{0.45\textwidth}
%    \begin{center}
%        \includegraphics[width=0.29\textwidth]{manometer}
%    \end{center}
%    \caption{Aneroid sphygmomanometer with stethoscope (Medicalexpo, 2012).}
%    \label{label:manometer}
%\end{wrapfigure}

%\begin{table}[!ht]\footnotesize
%    \centering
%    \begin{tabular}{cccccc}
%    \toprule
%    \multicolumn{2}{c} {Pearson's correlation test} & \multicolumn{4}{c} {Independent t-test} \\
%    \midrule    
%    \multicolumn{2}{c} {Gender} & \multicolumn{2}{c} {Activity level} & \multicolumn{2}{c} {Gender} \\
%    \midrule
%    Males & Females & 1st level & 6th level & Males & Females \\
%    \midrule
%    \multicolumn{2}{c} {BMI vs. SP} & \multicolumn{2}{c} {Systolic pressure} & \multicolumn{2}{c} {Systolic Pressure} \\
%    \multicolumn{2}{c} {BMI vs. DP} & \multicolumn{2}{c} {Diastolic pressure} & \multicolumn{2}{c} {Diastolic pressure} \\
%    \multicolumn{2}{c} {BMI vs. MAP} & \multicolumn{2}{c} {MAP} & \multicolumn{2}{c} {MAP} \\
%    \multicolumn{2}{c} {W:H ratio vs. SP} & \multicolumn{2}{c} {BMI} & \multicolumn{2}{c} {BMI} \\
%    \multicolumn{2}{c} {W:H ratio vs. DP} & \multicolumn{2}{c} {W:H ratio} & \multicolumn{2}{c} {W:H ratio} \\
%    \multicolumn{2}{c} {W:H ratio vs. MAP} & \multicolumn{2}{c} {\% Body fat} & \multicolumn{2}{c} {\% Body fat} \\
%    \multicolumn{2}{c} {} & \multicolumn{2}{c} {Height} & \multicolumn{2}{c} {Height} \\
%    \multicolumn{2}{c} {} & \multicolumn{2}{c} {Weight} & \multicolumn{2}{c} {Weight} \\
%    \multicolumn{2}{c} {} & \multicolumn{2}{c} {Heart rate} & \multicolumn{2}{c} {Heart rate} \\
%    \bottomrule
%    \end{tabular}
%    \caption{Parameters that were analysed and related statistical test performed for current study. BMI - body mass index; SP - systolic pressure; DP - diastolic pressure; MAP - mean arterial pressure; W:H ratio - waist to hip ratio.}
%    \label{label:tests}
%\end{table}
